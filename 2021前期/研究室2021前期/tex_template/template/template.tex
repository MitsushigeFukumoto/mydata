\documentclass[11pt,a4paper]{jsarticle}
\usepackage[top=30truemm,bottom=30truemm,left=20truemm,right=20truemm]{geometry}
\usepackage{fancybox}
\usepackage{fancyhdr}
\usepackage{lastpage}
\usepackage{listings,jlisting}
\usepackage{url}
\usepackage{color}
\usepackage{xcolor}
% graphicxは最後に読み込む
\usepackage[dvipdfmx]{graphicx}

\lstset{
language={Python},
backgroundcolor={\color[gray]{.85}},
basicstyle={\small},
identifierstyle={\small},
commentstyle={\small\ttfamily \color[rgb]{0,0.5,0}},
keywordstyle={\small\bfseries \color[rgb]{1,0,0}},
ndkeywordstyle={\small},
stringstyle={\small\ttfamily \color[rgb]{0,0,1}},
frame={tb},
breaklines=true,
columns=[l]{fullflexible},
numbers=left,
xrightmargin=0zw,
xleftmargin=3zw,
numberstyle={\scriptsize},
stepnumber=1,
numbersep=1zw,
captionpos=b,
morecomment=[l]{//}
}

\lstset{
language={TeX},
backgroundcolor={\color[gray]{.85}},
basicstyle={\small},
identifierstyle={\small},
commentstyle={\small\ttfamily \color[rgb]{0,0.5,0}},
keywordstyle={\small\bfseries \color[rgb]{1,0,0}},
ndkeywordstyle={\small},
stringstyle={\small\ttfamily \color[rgb]{0,0,1}},
frame={tb},
breaklines=true,
columns=[l]{fullflexible},
numbers=none,
xrightmargin=0zw,
xleftmargin=3zw,
numberstyle={\scriptsize},
stepnumber=1,
numbersep=1zw,
captionpos=b,
morecomment=[l]{//}
}

% listingにjsonを追加
\definecolor{delim}{RGB}{20,105,176}
\definecolor{numb}{RGB}{106, 109, 32}
\definecolor{string}{rgb}{0.64,0.08,0.08}

\lstdefinelanguage{json}{
    numbers=left,
    numberstyle=\small,
    frame=single,
    rulecolor=\color{black},
    showspaces=false,
    showtabs=false,
    breaklines=true,
    postbreak=\raisebox{0ex}[0ex][0ex]{\ensuremath{\color{gray}\hookrightarrow\space}},
    breakatwhitespace=true,
    basicstyle=\ttfamily\small,
    upquote=true,
    morestring=[b]",
    stringstyle=\color{string},
    literate=
     *{0}{{{\color{numb}0}}}{1}
      {1}{{{\color{numb}1}}}{1}
      {2}{{{\color{numb}2}}}{1}
      {3}{{{\color{numb}3}}}{1}
      {4}{{{\color{numb}4}}}{1}
      {5}{{{\color{numb}5}}}{1}
      {6}{{{\color{numb}6}}}{1}
      {7}{{{\color{numb}7}}}{1}
      {8}{{{\color{numb}8}}}{1}
      {9}{{{\color{numb}9}}}{1}
      {\{}{{{\color{delim}{\{}}}}{1}
      {\}}{{{\color{delim}{\}}}}}{1}
      {[}{{{\color{delim}{[}}}}{1}
      {]}{{{\color{delim}{]}}}}{1},
}

\def\title{議事録}
\def\date{2019/05/24}
\def\author{著者}

\fancypagestyle{mypagestyle}{%
\lhead{\title}%ヘッダ左を空に
\rhead{\date~~~~\author}%ヘッダ右を空に
\cfoot{- \thepage/\pageref{LastPage}\ -}%フッタ中央に"今のページ数/総ページ数"を設定
%\renewcommand{\headrulewidth}{0.0pt}%ヘッダの線を消す
}
\pagestyle{mypagestyle}

\begin{document}
\noindent\begin{center}
\doublebox{
\begin{minipage}{0.97\textwidth}
\begin{center}
\vspace{1em}
{\LARGE{\title}}
\vspace{1em}
\end{center}
\end{minipage}
} 
\end{center}

\section{\TeX について}
書き方はネットや本\cite{TeX}を参照.

\section{図の挿入}
図\ref{fig:sample_fig}を挿入する場合は次のように書く.
\begin{lstlisting}[language = TeX]
\begin{figure}[h]
    \begin{center}
        \includegraphics[width=0.9\textwidth]{AIT_logo.png}
        \caption{図のキャプションは下}
        % labelはcaptionの直後に書く
        \label{fig:sample_fig}
    \end{center}
\end{figure}
\end{lstlisting}

\begin{figure}[h]
    \begin{center}
        \includegraphics[width=0.9\textwidth]{AIT_logo.png}
        \caption{図のキャプションは下}
        % labelはcaptionの直後に書く
        \label{fig:sample_fig}
    \end{center}
\end{figure}

\newpage

\section{表の挿入}
表\ref{table:sample_table}を入れる参考として物理のかぎしっぽ\cite{buturi}を挙げる.

\begin{lstlisting}[language = TeX]
\begin{table}[h]
    \begin{center}
    \caption{表のキャプションは上}
    % labelはcaptionの直後に書く
    \label{table:sample_table}
    \begin{tabular}{|c||c|c|c|} \hline
        イオン粒子 &エネルギー &最大取り出し電流 \\ \hline\hline
        陽子       &$17.0\,\mathrm{MeV}$ &$50\,\mathrm{\mu A}$ \\ \hline
                &$4.25\,\mathrm{MeV}$ &$50\,\mathrm{\mu A}$ \\ \hline
        重陽子     &$10.0\,\mathrm{MeV}$ &$50\,\mathrm{\mu A}$ \\ \hline
        ${}^4$He   &$20.0\,\mathrm{MeV}$ &$20\,\mathrm{\mu A}$ \\ \hline
        ${}^3$He   &$26.0\,\mathrm{MeV}$ &$20\,\mathrm{\mu A}$ \\ \hline
    \end{tabular}
    \end{center}
\end{table}
\end{lstlisting}
\begin{table}[h]
    \begin{center}
    \caption{表のキャプションは上}
    % labelはcaptionの直後に書く
    \label{table:sample_table}
    \begin{tabular}{|c||c|c|c|} \hline
        イオン粒子 &エネルギー &最大取り出し電流 \\ \hline\hline
        陽子       &$17.0\,\mathrm{MeV}$ &$50\,\mathrm{\mu A}$ \\ \hline
                &$4.25\,\mathrm{MeV}$ &$50\,\mathrm{\mu A}$ \\ \hline
        重陽子     &$10.0\,\mathrm{MeV}$ &$50\,\mathrm{\mu A}$ \\ \hline
        ${}^4$He   &$20.0\,\mathrm{MeV}$ &$20\,\mathrm{\mu A}$ \\ \hline
        ${}^3$He   &$26.0\,\mathrm{MeV}$ &$20\,\mathrm{\mu A}$ \\ \hline
    \end{tabular}
    \end{center}
\end{table}

\subsection{参考文献の挿入}
TeXの使い方\cite{TeX}のように参考文献を挿入する方法は下記の通り.
\begin{lstlisting}[language = TeX]
TeXの使い方\cite{TeX}
\end{lstlisting}

\section{箇条書き}
箇条書きの例は次の通り.
\begin{lstlisting}[language = TeX]
\begin{itemize}
\item{議事録担当(    )}
\item{話題提供(    )}
\item{報告}
\item{セミナー}
\item{次回}
\end{itemize}
\end{lstlisting}

\begin{itemize}
\item{議事録担当(    )}
\item{話題提供(    )}
\item{報告}
\item{セミナー}
\item{次回}
\end{itemize}
%

\section{プログラムソースを書く}
Pythonのプログラムソースを書く場合は次のように書く.
\begin{lstlisting}[language = TeX]
\begin{lstlisting}[language = Python]
#/usr/bin/env python
h = {}
h['hoge'] = 1
h['fuga'] = 2
print(h['hoge']+h['fuga'])
\end{lstlisting}
\begin{lstlisting}[language = Python]
#/usr/bin/env python
h = {}
h['hoge'] = 1
h['fuga'] = 2
print(h['hoge']+h['fuga'])
\end{lstlisting}

\section{参考文献の挿入について}
bibtexを利用する場合は,ptex2pdfを実行する前にpbibtexを実行する.
vscodeのTEXエクステンションで`pbibtex + ptex2pdf`を選択すること.
\begin{lstlisting}[language = json]
{
    "name": "pbibtex+ptex2pdf",
    "tools": [
         "Step 1: ptex2pdf",
         "Step 2: pbibtex",
         "Step 3: ptex2pdf",
         "Step 4: ptex2pdf"
    ]
}
\end{lstlisting}

% jplainだとアルファベット順になる
\bibliographystyle{junsrt}

\bibliography{template}
\end{document}